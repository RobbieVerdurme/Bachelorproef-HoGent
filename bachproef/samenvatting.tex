%%=============================================================================
%% Samenvatting
%%=============================================================================

% TODO: De "abstract" of samenvatting is een kernachtige (~ 1 blz. voor een
% thesis) synthese van het document.
%
% Deze aspecten moeten zeker aan bod komen:
% - Context: waarom is dit werk belangrijk?
% - Nood: waarom moest dit onderzocht worden?
% - Taak: wat heb je precies gedaan?
% - Object: wat staat in dit document geschreven?
% - Resultaat: wat was het resultaat?
% - Conclusie: wat is/zijn de belangrijkste conclusie(s)?
% - Perspectief: blijven er nog vragen open die in de toekomst nog kunnen
%    onderzocht worden? Wat is een mogelijk vervolg voor jouw onderzoek?
%
% LET OP! Een samenvatting is GEEN voorwoord!

%%---------- Nederlandse samenvatting -----------------------------------------
%
% TODO: Als je je bachelorproef in het Engels schrijft, moet je eerst een
% Nederlandse samenvatting invoegen. Haal daarvoor onderstaande code uit
% commentaar.
% Wie zijn bachelorproef in het Nederlands schrijft, kan dit negeren, de inhoud
% wordt niet in het document ingevoegd.

\IfLanguageName{english}{%
\selectlanguage{dutch}
\chapter*{Samenvatting}
\selectlanguage{english}
}{}

%%---------- Samenvatting -----------------------------------------------------
% De samenvatting in de hoofdtaal van het document

\chapter*{\IfLanguageName{dutch}{Samenvatting}{Abstract}}
In dit onderzoek wordt er een antwoord gegeven op de vraag of de PWA de native applicatie zal kunnen vervangen, dit wordt getest aan de hand van verschillende onderzoeksvragen.

Voor we deze onderzoeken kunnen uitvoeren moet er eerst informatie verzameld worden over de PWA en zijn functies. Bij een PWA is de installatie belangrijk punt. Voordat een PWA kan geïnstalleerd worden moet de browser een aantal PWA functionaliteiten ondersteunen zoals een service worker en een manifest bestand, deze zijn essentieel voor een PWA. Om de service worker te configureren is er een strategie nodig over hoe de service worker de verschillende netwerk aanvragen zal moeten behandelen, er zijn 5 verschillende manieren om dit te doen. De standaard strategie maakt gebruik van de stale-while-revalidate. Dit is een strategie die de data zoveel mogelijk online ophaalt indien het apparaat een internet connectie heeft. Indien dit niet zo is zal de stale-while-revalidate strategie de cache erop nakijken of deze netwerk aanvraag al reeds gebeurt is en het opgeslagen antwoord ervan teruggeven.

Om een PWA applicatie te ontwikkelen zijn er verschillende frameworks die in aanmerking komen. De bekendste frameworks hebben een pakket ontwikkeld dat kan toegevoegd worden in het project en automatisch een PWA genereert zonder veel extra configuratie. Dit is handig omdat de developer zich dan kan focussen op het maken van de functionaliteiten van de PWA. Zo een PWA pakket zal een service worker en een manifest bestand genereren eens de applicatie gebouwd wordt, de developer kan ervoor kiezen om dit te overschrijven. 

In deze bachelorproef zijn verschillende onderzoeksvragen beantwoord. Eén van deze vragen onderzocht de voordelen van de PWA versus de native applicatie. Een eerste voordeel is dat een PWA tot 95 % kleiner is dan een native applicatie, dit omdat de PWA de styling en data bij elke pagina opnieuw ophaalt eens er internetverbinding is. Om deze reden moet de PWA een minimaal aan data installeren op het apparaat. Een ander voordeel van de PWA is dat deze maar één codebase nodig heeft voor zowel IOS, Android en Web. Hierdoor is het onderhoud van een PWA project veel gemakkelijker.
Een andere onderzoeksvraag focust zich op de gebruiksvriendelijkheid en toegankelijkheid van de applicaties. Hieruit bleek dat de PWA niet zelfstandig geïnstalleerd kon worden door de proefpersonen omdat deze niet in de store te verkrijgen was zoals de native applicatie. Er waren zowel voor- als nadelen aan de PWA op vlak van gebruiksvriendelijkheid, onder andere dat de filter bij de PWA niet in het oog sprong en dus onopgemerkt bleef door de verschillende proefpersonen. 

Op basis van dit onderzoek kan er geargumenteerd worden dat de PWA op verschillende vlakken nog niet klaar is om de native applicatie te vervangen. Verschillende pakketten zijn nog niet beschikbaar zoals het aanspreken van de bluetooth module, het aanspreken van andere applicaties, enzovoort. In combinatie met de limitaties die de PWA heeft om in de play- en appstore gepubliceerd te worden, kunnen we concluderen dat de native applicatie waarschijnlijk nog even gebruikt zal worden als standaard voor applicatie ontwikkeling. Eens deze problemen opgelost worden kan de PWA een waardige vervanger zijn voor de native applicatie.


