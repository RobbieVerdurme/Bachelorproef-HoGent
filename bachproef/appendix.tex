\appendix 
%\addcontentsline{toc}{chapter}{Code PWA}
\chapter{Code PWA}
De volledige code van de menu applicatie PWA is beschikbaar op https://github.com/RobbieVerdurme/MenuApp\_PWA.

\section{index.vue}
\begin{lstlisting}[caption=index.vue, language=Javascript]
<template>
<picker class="app-content" />
</template>

<script>
export default {
middleware: 'data',
components: {
picker: () => import('~/components/molecules/picker')
}
}
</script>
<style scoped>
.app-content {
min-height: 80vh;
}
</style>
\end{lstlisting}


\section{picker.vue}
\begin{lstlisting}[caption=picker.vue, language=Javascript]
<template>
<div class="center">
<!--Selected item-->
<nuxt-link v-if="selectedMenu.key" :to="{name: 'menu-id-info', params: {id: selectedMenu.key}}">
<span class="md-display-1">{{ selectedMenu.name }}</span>
</nuxt-link>
<span v-else class="md-display-1">{{ selectedMenu.name }}</span>
<!--generate button-->
<md-button class=" md-fab md-raised md-primary" @click="generateRandomMenuitem">
<md-icon>cached</md-icon>
</md-button>
</div>
</template>

<script>
export default {
data () {
return {
selectedMenu: { name: 'Click on the button to generate a random menu' }
}
},
methods: {
/**
* generate a random menu item
*/
generateRandomMenuitem () {
const menulistLength = this.$store.getters.getMenuListLength
if (!menulistLength) {
this.selectedMenu = { title: 'The menulist is empty' }
} else {
const randomnumber = Math.floor(Math.random() * menulistLength)
this.selectedMenu = this.$store.getters.getMenuitem(randomnumber)
}
}
}
}
</script>

<style scoped>
.center span {
text-align: center;
margin: 0;
position: absolute;
top: 40%;
left: 50%;
margin-right: -50%;
transform: translate(-50%, -50%)
}

.center button {
margin: 0;
position: absolute;
top: 70%;
left: 50%;
margin-right: -50%;
transform: translate(-50%, -50%)
}
</style>

\end{lstlisting}


\section{getters.js}
Dit is een bestand van de vuex store.
\begin{lstlisting}[caption= store getters.js, language= Javascript]
/**
* getters
*/
export default {
/** *******************USER********************** */
/**
* get loggedin user
*/
getLogin: (state) => {
return state.loggedin
},

/** *******************MENU********************** */
/**
* get selected menu
*/
getSelectedMenu: (state) => {
return state.selectedMenu
},

/**
* get menu item on place number
*/
getMenuitem: state => (index) => {
return state.menus[index]
},

/**
* get menu item with id
*/
getMenuitemWithId: state => (id) => {
return state.menus.find(m => m.key === id)
},

/**
* get the max length of the menulist
*/
getMenuListLength: state => state.menus.length,

/**
* get all menu items
*/
getAllMenuItems: state => state.menus,

/**
* get menus with filter text
*/
getAllMenuWithFilter: state => (filtertext) => {
return state.menus.filter(m => m.name.toLowerCase().includes(filtertext.toLowerCase()) || m.ingredients.find(i => i.name.toLowerCase().includes(filtertext.toLowerCase())))
}
}

\end{lstlisting}

\section{state.js}
Dit is een bestand van de vuex store

\begin{lstlisting}[caption= store state.js ,language=Javascript]
/**
* state
*/
export default () => ({
menus: [],
selectedMenu: {},
loggedin: false
})

\end{lstlisting}

% \addcontentsline{toc}{chapter}{Code Native Applicatie}
\chapter{Code Native Applicatie}
De volledige code van de menu applicatie native applicatie is beschikbaar op https://github.com/RobbieVerdurme/MenuAppV2

\section{MainActivity}
\begin{lstlisting}[caption= Mainactivity.kt , language=Kotlin]
package com.example.menuappv2.ui.activity

import androidx.appcompat.app.AppCompatActivity
import android.os.Bundle
import com.example.menuappv2.R
import com.google.firebase.FirebaseApp

class MainActivity : AppCompatActivity() {

override fun onCreate(savedInstanceState: Bundle?) {
super.onCreate(savedInstanceState)
FirebaseApp.initializeApp(this)
setContentView(R.layout.activity_main)
//Setup the toolbar, we need it throughout the app.
setSupportActionBar(this.findViewById(R.id.mainActivityToolbar))
//We need to hide the action bar in the splash screen.
supportActionBar?.hide()
}
}
\end{lstlisting}

\section{MenuDecideFragment}
\begin{lstlisting}[caption= MenuDecideFragment.kt , language=Kotlin]
package com.example.menuappv2.ui.fragment

import android.os.Bundle
import android.view.LayoutInflater
import android.view.View
import android.view.ViewGroup
import androidx.fragment.app.Fragment
import androidx.navigation.fragment.findNavController
import com.example.menuappv2.R
import com.example.menuappv2.databinding.FragmentMenuDecideBinding
import com.example.menuappv2.viewmodel.DecideViewModel
import com.example.menuappv2.viewmodel.MenuDetailViewModel
import kotlinx.android.synthetic.main.fragment_menu_decide.*
import org.koin.androidx.viewmodel.ext.android.sharedViewModel
import org.koin.androidx.viewmodel.ext.android.viewModel

class MenuDecideFragment: Fragment() {
/**
* The [DecideViewModel] for this fragment.
*/
val viewModel: DecideViewModel by viewModel()
val detailViewModel: MenuDetailViewModel by sharedViewModel()

override fun onCreateView(inflater: LayoutInflater, container: ViewGroup?, savedInstanceState: Bundle?): View? {
val binding = FragmentMenuDecideBinding.inflate(inflater, container, false)
binding.lifecycleOwner = this
binding.viewModel = viewModel
return binding.root
}

override fun onViewCreated(view: View, savedInstanceState: Bundle?) {
super.onViewCreated(view, savedInstanceState)
setupFragment()
}

fun setupFragment() {
lblMenu.setOnClickListener {
if(!lblMenu.text.contains("decide")){
detailViewModel.setMenu(viewModel.getMenu())
findNavController().navigate(R.id.menuDetailFragment)
}
}
}
}
\end{lstlisting}

\section{MenuRepository}
\begin{lstlisting}[caption= MenuRepository.kt , language=Kotlin]
package com.example.menuappv2.network

import android.widget.Toast
import androidx.lifecycle.LiveData
import androidx.lifecycle.MutableLiveData
import com.example.menuappv2.model.Food
import com.google.firebase.database.DataSnapshot
import com.google.firebase.database.DatabaseError
import com.google.firebase.database.ValueEventListener
import com.google.firebase.database.ktx.database
import com.google.firebase.ktx.Firebase
import java.util.ArrayList

class MenuRepository {
private var liveFoodList :MutableLiveData<ArrayList<Food>> = MutableLiveData<ArrayList<Food>>(arrayListOf())
private var foodList = ArrayList<Food>()
private val firebaseDatabase = Firebase.database
private val databaseRefrenceData = firebaseDatabase.getReference("FoodList")

init {
databaseRefrenceData.keepSynced(true)

databaseRefrenceData.addValueEventListener(object : ValueEventListener {
override fun onCancelled(p0: DatabaseError) {
println("Failed to read value ${p0.message}")
}

override fun onDataChange(dataSnapshot: DataSnapshot) {
foodList.clear()
for (h in dataSnapshot.child("Food").children){
val food = h.getValue(Food::class.java)
food?.setKey(h.key!!)
foodList.add(food!!)
}
liveFoodList.value!!.clear()
liveFoodList.value!!.addAll(foodList)
}
})
}

fun getFoodList(): LiveData<List<Food>> {
return liveFoodList as LiveData<List<Food>>
}

fun remove(food:Food): Boolean{
databaseRefrenceData.child("Food").child(food.getKey()).removeValue()
liveFoodList.value!!.remove(food)
return true
}

fun save(food : Food): Boolean{
if(food.getKey().isEmpty()){
val key = databaseRefrenceData.child("Food").push().key
if(key != null){
food.setKey(key)
databaseRefrenceData.child("Food").push().setValue(food)
liveFoodList.value!!.add(food)
return true
}
}else{
databaseRefrenceData.child("Food").child(food.getKey()).setValue(food)
val index = liveFoodList.value!!.indexOf(food)
liveFoodList.value!![index] = food
return true
}
return false
}
}
\end{lstlisting}

\section{fragment\_menu\_decide.xml}
\begin{lstlisting}[caption= fragment\_menu\_decide.xml]
<?xml version="1.0" encoding="utf-8"?>
<layout xmlns:android="http://schemas.android.com/apk/res/android"
xmlns:app="http://schemas.android.com/apk/res-auto"
xmlns:tools="http://schemas.android.com/tools">
<data>
<variable name="viewModel" type="com.example.menuappv2.viewmodel.DecideViewModel"/>
</data>
<androidx.constraintlayout.widget.ConstraintLayout
android:layout_width="match_parent"
android:layout_height="match_parent">

<TextView
android:text="@{viewModel.chosenRandomfoodName}"
android:layout_width="0dp"
android:layout_height="wrap_content"
android:id="@+id/lblMenu"
app:layout_constraintTop_toTopOf="parent"
app:layout_constraintEnd_toEndOf="parent"
app:layout_constraintStart_toStartOf="parent"
app:layout_constraintBottom_toTopOf="@+id/btnDecide"
app:layout_constraintHorizontal_bias="0.506"
app:layout_constraintVertical_bias="0.431"
android:textAppearance="@style/TextAppearance.AppCompat.Medium"
android:textSize="30sp"
android:textAlignment="center"/>
<Button
android:text="@string/decide"
android:onClick="@{() -> viewModel.RandomFood()}"
android:layout_width="0dp"
android:layout_height="95dp"
android:id="@+id/btnDecide"
app:layout_constraintBottom_toBottomOf="parent" app:layout_constraintEnd_toEndOf="parent"
app:layout_constraintStart_toStartOf="parent"
/>
</androidx.constraintlayout.widget.ConstraintLayout>
</layout>
\end{lstlisting}
