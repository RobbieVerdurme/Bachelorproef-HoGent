%%=============================================================================
%% Voorwoord
%%=============================================================================

\chapter*{\IfLanguageName{dutch}{Woord vooraf}{Preface}}
\label{ch:voorwoord}

%% TODO:
%% Het voorwoord is het enige deel van de bachelorproef waar je vanuit je
%% eigen standpunt (``ik-vorm'') mag schrijven. Je kan hier bv. motiveren
%% waarom jij het onderwerp wil bespreken.
%% Vergeet ook niet te bedanken wie je geholpen/gesteund/... heeft
In deze bachelorproef wordt er een vergelijking gemaakt tussen een native applicatie en een progressive web applicatie. De reden hiervoor is mijn stageplaats bij Digipolis. Digipolis wil graag onderzoeken wat de mogelijkheden van een progressive web applicatie zijn, zodat deze technologie eventueel gebruikt kan worden binnen het bedrijf. De reden waarom digipolis dit wil onderzoeken is om te weten te komen of applicaties een meerwaarde kunnen bieden voor de doelstellingen van digipolis (burgers informeren en helpen participeren) en welke methode best past binnen hun ecosysteem.

Graag wil ik mijn promotor Lieven Smits bedanken voor de feedback en de steun tijdens het schrijven van de bachelorproef. Daarnaast wil ik ook Bart Delrue bedanken voor de technische feedback en de begeleiding tijdens het verloop van de bachelorproef.





