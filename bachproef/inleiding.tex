%%=============================================================================
%% Inleiding
%%=============================================================================

\chapter{\IfLanguageName{dutch}{Inleiding}{Introduction}}
\label{ch:inleiding}

%De inleiding moet de lezer net genoeg informatie verschaffen om het onderwerp te begrijpen en in te zien waarom de onderzoeksvraag de moeite waard is om te onderzoeken. In de inleiding ga je literatuurverwijzingen beperken, zodat de tekst vlot leesbaar blijft. Je kan de inleiding verder onderverdelen in secties als dit de tekst verduidelijkt. Zaken die aan bod kunnen komen in de inleiding~\autocite{Pollefliet2011}:

%\begin{itemize}
%  \item context, achtergrond
%  \item afbakenen van het onderwerp
%  \item verantwoording van het onderwerp, methodologie
%  \item probleemstelling
%  \item onderzoeksdoelstelling
%  \item onderzoeksvraag
%  \item \ldots
%\end{itemize}

\section{\IfLanguageName{dutch}{Probleemstelling}{Problem Statement}}
\label{sec:probleemstelling}

Digipolis ontwikkelt verschillende webomgevingen voor de stad Gent. Voorbeelden hiervan zijn de website van de stad Gent (https://stad.gent) en visit Gent (https://visit.gent.be). Graag wil digipolis weten of een applicatie van stad Gent handig zou zijn voor de bezoeker, zodat deze de site niet meer moet opzoeken. Daarnaast heeft een applicatie heeft het voordeel van offline beschikbaar te zijn. Hierdoor kan de bezoeker zonder enige data te verbruiken de applicatie toch raadplegen.

Voor dit probleem kan een progressive web applicatie de oplossing bieden. Een progressive web applicatie is een applicatiesoftware die via het web wordt geleverd en is gebouwd met veelgebruikte web technologieën zoals HTML, CSS en JavaScript. (\cite{DEFINITION_PWA}). Door deze technologie is er maar één codebase nodig voor de verschillende platformen(Android, IOS, Web) en moeten de aanpassingen en het onderhoud maar op één plaats toegepast worden.

De native applicatie is een andere oplossing voor dit probleem. Een native applicatie wordt specifiek ontwikkeld voor een platform (Android, iOS, Windows Phone) in een eigen codeertaal (\cite{DIFF_NATIVEAPP_PWA}). Deze applicaties zijn enkel verkrijgbaar in de app store of play store. De applicatie zal niet beschikbaar zijn via de browser.

Vervolgens wil men weten welke van deze twee oplossingen beter is op vlak van performance en benodigde ruimte. Bijkomend is het belangrijk om de applicatie snel terug te vinden in een vertrouwde omgeving en dat de applicatie er vertrouwd uitziet om de gebruikerservaring te garanderen.


\section{\IfLanguageName{dutch}{Onderzoeksvraag}{Research question}}
\label{sec:onderzoeksvraag}
Dit onderzoek zal nagaan hoever de progressive web applicatie de native applicatie kan benaderen. 

Hiervoor zijn volgende onderzoeksvragen opgesteld:
\begin{itemize}
	\item Wat zijn de voordelen van PWA versus Native Apps?
	\item Welke frameworks komen hiervoor in aanmerking?
	\item  Wat is de impact van een PWA op de gebruikerservaring en toegankelijkheid?
	\item Kan de PWA de native app vervangen?
\end{itemize}

\section{\IfLanguageName{dutch}{Onderzoeksdoelstelling}{Research objective}}
\label{sec:onderzoeksdoelstelling}

De doelstelling van deze bachelorproef is een vergelijking maken tussen de native applicatie en een progressive web applicatie. Deze zullen getest worden op verschillende vlakken namelijk de performance, de benodigde ruimte en de gebruikerservaring. Uiteindelijk zal er een besluit gevormd worden waarin duidelijk zal gemaakt worden voor welke applicaties een progressive web applicatie de beste keuze is.

\section{\IfLanguageName{dutch}{Opzet van deze bachelorproef}{Structure of this bachelor thesis}}
\label{sec:opzet-bachelorproef}

% Het is gebruikelijk aan het einde van de inleiding een overzicht te
% geven van de opbouw van de rest van de tekst. Deze sectie bevat al een aanzet
% die je kan aanvullen/aanpassen in functie van je eigen tekst.

De rest van deze bachelorproef is als volgt opgebouwd:

In Hoofdstuk~\ref{ch:stand-van-zaken} wordt een overzicht gegeven van de stand van zaken binnen het onderzoeksdomein, op basis van een literatuurstudie.

% TODO: Vul hier aan voor je eigen hoofstukken, één of twee zinnen per hoofdstuk
In Hoofdstuk~\ref{ch:kotlin}, ~\ref{ch:vue} en ~\ref{ch:nuxt} worden de belangrijkste aspecten toegelicht van de gekozen frameworks

In Hoofdstuk~\ref{ch:methodologie} wordt de methodologie toegelicht en worden de gebruikte onderzoekstechnieken besproken om een antwoord te kunnen formuleren op de onderzoeksvragen.

In Hoofdstuk~\ref{ch:conclusie}, tenslotte, wordt de conclusie gegeven en een antwoord geformuleerd op de onderzoeksvragen. Daarbij wordt ook een aanzet gegeven voor toekomstig onderzoek binnen dit domein.